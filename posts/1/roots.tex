\documentclass[a4paper, 10pt]{article}

\usepackage{amsfonts}
\usepackage{amsmath}
\usepackage{amsthm}
\usepackage{biblatex}
\usepackage{fullpage}
\usepackage{hyperref}
\usepackage{mathtools}
\usepackage{mdframed}
\usepackage{wasysym}

\hypersetup{
    % See https://tex.stackexchange.com/a/51349
    colorlinks   = true, %Colours links instead of ugly boxes
    urlcolor     = blue, %Colour for external hyperlinks
    linkcolor    = blue, %Colour of internal links
    citecolor    = red   %Colour of citations
}

\addbibresource{roots.bib}

\theoremstyle{definition}
\newtheorem*{statement}{Statement}
\theoremstyle{definition}
\newtheorem*{definition}{Definition}

\setlength{\parindent}{0pt}

\begin{document}
    \title{
        \vspace{-5ex}
        \large Computing roots with interval nestings
    }
    \author{
        \normalsize valentinpi
    }
    \date{
        \normalsize \today\\
        (newest version)
    }
    \maketitle

    \begin{abstract}
        Computing roots of positive reals using a calculator is quite easy, but implementing it? In this paper, I discuss a quick way (not as in \emph{efficient}) of approximating roots of any real number. Then I present a small implementation in the Python programming language. Much of the work here is taken from \cite{analysiskoenigsberger}.
    \end{abstract}

    \begin{center}
        \huge
        BLOG PREVIEW, NOT FINISHED
    \end{center}
    
    \tableofcontents

    \section{Introduction}
    
    \subsection{Notation}
    
    I denote the set of natural numbers with \(\mathbb{N}\), the set of real numbers with \(\mathbb{R}\) and the set of positive reals with \(\mathbb{R}_+\). I also assume basic knowledge in set theory and notation, mathematical proofs and notation and fields.
    
    \section{Construction of roots}
    
    \subsection{Completeness of \texorpdfstring{\(\mathbb{R}\)}{R}}
    The real numbers \(\mathbb{R}\) are structured in three ways:
    \begin{itemize}
        \item Field structure through it's axioms and all derivable calculation rules
        \item Ordering of real numbers with the property of positivity
        \item Completeness
    \end{itemize}
    
    We will take a look at the latter. With the rational numbers, we are not able to describe every point in a unit line
    \[
        \{\,x \in \mathbb{R} \mid 0 \leq x \leq 1 \,\}
    \]
    One famous example for this is the reciprocal of the golden ratio. For the irrationality proof, consider reading \cite{analysiskoenigsberger}.

    There are multiple ways of introducing the completeness of the field \(\mathbb{R}\). \cite{analysiskoenigsberger} introduces this concept using so-called interval nestings.

    \begin{definition}
        Let \(a, b \in \mathbb{R}\) with \(a < b\). We define the \emph{closed interval}
        \[
            [a; b] \coloneqq \{\, x \in \mathbb{R} \mid a \leq x \leq b \,\}
        \]
        with the \emph{boundaries} \(a, b\). The number \(b - a = |[a; b]|\) is called \emph{length} of the interval. Closed intervals are also commonly known as \emph{compact}.
    \end{definition}

    This definition should be well-known. Now to interval nestings.

    \newcommand{\intv}{\text{I}}
    \begin{definition}
        An \emph{interval nesting} is a series of compact intervals \(\intv_1, \intv_2, ...\), short \((\intv_n)\) that satisfy the following two properties

        \item[\hspace{0.5cm}(I.1)] \(\intv_{n+1} \supset \intv_n\)
        
        \item[\hspace{0.5cm}(I.2)] \(\forall \, \varepsilon > 0 \, \exists \, n \in \mathbb{N}\colon |\intv_n| < \varepsilon\)
    \end{definition}

    The latter property (I.2) can be interpreted as that the intervals can get arbitrarily small. We can define interval nestings using induction.

    The completeness of \(\mathbb{R}\) is founded on the following statement.

    \begin{mdframed}[innertopmargin=0]
        \begin{statement}
            For every interval nesting \((\intv_n)\) in the \(\mathbb{R}\), there exists a real number \(s\), which is included in every interval.
        \end{statement}
    \end{mdframed}

    We assume the truth of this statement without proof. The statement itself does not exclude multiple numbers \(s\), but the following proof will rule that out.

    \begin{statement}
        The number \(s\) is unique.
    \end{statement}

    \begin{proof}
        Let's say two numbers \(s \neq t\) are included in every interval. Without loss of generality, say \(s < t\). Then every interval is of length \(\geq t - s\), which contradicts (I.2). \lightning
    \end{proof}

    \subsection{Existence of roots}

    With the introduction of interval nestings, we can now prove a theorem that directly constructs roots. Some prerequisites first, which we will use.

    %\begin{lemma}
        
    %\end{lemma}

    Now for the main part.

    \begin{statement}
        For every real number \(x > 0\) and every \(k \in \mathbb{N}\), there is one and only one real number \(y > 0\) with \(y^k = x\). We call it the \(k\)th root of \(x\), in symbols \(y = x^\frac{1}{x}, y = \sqrt[k]{x}\).
    \end{statement}

    \begin{proof} We consider the case \(x > 1\), since for \(x < 1\) we can make the transition \(x' \coloneqq 1/x\), which complies with the ordering of \(\mathbb{R}\). And for \(x = 1\) it holds that \(y = 1\).\\

    First, we construct an interval nesting \((I_n)\) in the \(\mathbb{R}_+\).
    
    We begin with the properties our nesting should hold. For every interval \(I_n = [a_n; b_n], n \in \mathbb{N}\), the following should hold\\

    \item{\hspace{0.5cm}(\(1_n\))} \(a_n^k \leq x \leq b_n^k\)
    
    \item{\hspace{0.5cm}(\(2_n\))} \(|\intv_n| = \left(\frac{1}{2}\right)^{n-1}\cdot|\intv_1|\)\\

    So every interval interval \(\intv_{n + 1}\) is half the length of the previous interval \(\intv_n\) and we can already see that the number included in every interval is \(x\).\\

    Now for the construction we use induction. We declare the first interval and then, by the properties of the natural numbers, declare the next ones.

    Let \(\intv_1 \coloneqq [1; x]\). We verify that the properties (\(1_1\)) and (\(2_1\)) hold.
    
    Now let \(n\) be arbitrary, but fixed, with \(\intv_n = [a_n; b_n]\) holding the properties (\(1_n\)) and (\(2_n\)).
    
    We construct the next interval \(n + 1\) by cutting the \(n\)th one in half. Let \(m \coloneqq \frac{b_n - a_n}{2}\) be the center of the interval. We then define
    \[
        \intv_{n+1} \coloneqq \begin{cases}
            [a_n; m] & m \geq x\\
            [m; b_n] & m < x
        \end{cases}
    \]

    Due to our construction, (\(1_{n+1}\)) and (\(2_{n+1}\)) both hold, since
    
    \[
        \begin{cases}
            a_{n+1} = a_n \leq x \leq m < b_n = b_{n+1} \Rightarrow [a_{n+1};b_{n+1}] \supset [a_n;b_n] & m \geq x\\
            a_n < a_{n+1} = m < x \leq b_n = b_{n+1} \Rightarrow [a_{n+1};b_{n+1}] \supset [a_n;b_n] & m < x
        \end{cases}
    \]

    and
    
    \[
        |\intv_{n+1}| = \frac{1}{2} \cdot |I_n| = \frac{1}{2} \cdot \left(\frac{1}{2}\right)^{n-1} \cdot |I_1| = \left(\frac{1}{2}\right)^{n+1-1} \cdot |I_1|
    \]

    We must now still prove that both (I.1) and (I.2) hold.

    \item{\hspace{0.5cm}(I.1)} See above. % TODO: Add refs
    
    \item{\hspace{0.5cm}(I.2)} With \((2_n)\) consider
    \[
        \left(\frac{1}{2}\right)^{n-1=n'} < \varepsilon' = \varepsilon \cdot |\intv_1|^{-1}
    \]
    With the lemma we have proven above, there exists such an integer \(n'\) and therefore
    \[
        \left(\frac{1}{2}\right)^{n-1} \cdot |\intv_1| < \varepsilon 
    \]
    which satisfies the property.

    Let \(y\) be the number that is included  in all intervals \((\intv_n)\). We will now show that \(y^k = x\).

    \end{proof}

    \section{Implementation}

    \printbibliography
\end{document}
    