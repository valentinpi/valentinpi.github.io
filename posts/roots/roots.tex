\documentclass[a4paper, 10pt]{article}

\usepackage{amsfonts}
\usepackage{amsmath}
\usepackage{amsthm}
\usepackage{biblatex}
\usepackage{enumitem}
\usepackage{fullpage}
\usepackage{hyperref}
\usepackage{mathtools}
\usepackage{mdframed}
\usepackage{wasysym}

\hypersetup{
    % See https://tex.stackexchange.com/a/51349
    colorlinks   = true, %Colours links instead of ugly boxes
    urlcolor     = blue, %Colour for external hyperlinks
    linkcolor    = blue, %Colour of internal links
    citecolor    = red   %Colour of citations
}

\addbibresource{roots.bib}

\theoremstyle{remark}
\newtheorem{lemma}{Lemma}

\theoremstyle{definition}
\newtheorem{definition}{Definition}
\newtheorem{theorem}{Theorem}
\newtheorem*{theorem*}{Theorem}

\setlength{\parindent}{0pt}

\begin{document}
    \title{
        \vspace{-5ex}
        \large Computing roots with interval nestings
    }
    \author{
        \normalsize valentinpi
    }
    \date{
        \normalsize \today\\
        (newest version)
    }
    \maketitle

    \begin{abstract}
        Computing roots of positive reals using a calculator is quite easy, but implementing it? In this paper, I discuss a quick way (not as in \emph{efficient}) of approximating roots of any real number. Then I present a small implementation in the Python programming language. Much of the work here is taken from \cite{analysiskoenigsberger}.
    \end{abstract}

    \begin{center}
        \huge
        BLOG PREVIEW, NOT FINISHED
    \end{center}
    
    \tableofcontents

    \section{Introduction}
    
    \subsection{Notation}
    
    I denote the set of natural numbers with \(\mathbb{N}\), the set of real numbers with \(\mathbb{R}\) and the set of positive reals with \(\mathbb{R}_+\). I also assume basic knowledge in set theory and notation, mathematical proofs and notation and fields.\\

    There are a few things which need to be known about the field \(\mathbb{R}\), but I will try to keep this paper short and only mention the most important ones for the most important proofs.
    
    \section{Construction of roots}
    
    \subsection{Completeness of \texorpdfstring{\(\mathbb{R}\)}{R}}
    The real numbers \(\mathbb{R}\) are structured in three ways:
    \begin{itemize}
        \item Field structure through it's axioms and all derivable calculation rules
        \item Ordering of real numbers with the property of positivity
        \item Completeness
    \end{itemize}
    
    We will take a look at the latter. With the rational numbers, we are not able to describe every point in a unit line
    \[
        \{\,x \in \mathbb{R} \mid 0 \leq x \leq 1 \,\}
    \]
    One famous example for this is the reciprocal of the golden ratio. For the irrationality proof, consider reading \cite{analysiskoenigsberger}. While the rational numbers therefore are, in a sense, incomplete, the real numbers do not have this issue.

    There are multiple ways of introducing the completeness of the field \(\mathbb{R}\). \cite{analysiskoenigsberger} introduces this concept, for one, using so-called interval nestings.

    \begin{definition}
        Let \(a, b \in \mathbb{R}\) with \(a < b\). We define the \emph{closed interval}
        \[
            [a; b] \coloneqq \{\, x \in \mathbb{R} \mid a \leq x \leq b \,\}
        \]
        with the \emph{boundaries} \(a, b\). The number \(b - a \eqqcolon |[a; b]|\) is called \emph{length} of the interval. Closed intervals are also commonly known as \emph{compact}.
    \end{definition}

    This definition should be well-known. Now to interval nestings.

    \newcommand{\intv}{\text{I}}
    \begin{definition}
        An \emph{interval nesting} is a series of compact intervals \(\intv_1, \intv_2, ...\), short \((\intv_n)\), that satisfy the following two properties

        \begin{itemize}[label={}]
            \item[(I.1)] \(\intv_{n+1} \supset \intv_n\)
            
            \item[(I.2)] \(\forall \, \varepsilon > 0 \, \exists \, n \in \mathbb{N}\colon |\intv_n| < \varepsilon\)
        \end{itemize}

    \end{definition}

    The latter property (I.2) can be interpreted as that the intervals can get arbitrarily small. We can define interval nestings using induction.

    The completeness of \(\mathbb{R}\) is founded on the following theorem.

    \begin{mdframed}[innertopmargin=0]
        \begin{theorem*}[Principle of Interval Nesting]
            For every interval nesting \((\intv_n)\) in the \(\mathbb{R}\), there exists a real number \(s\), which is included in every interval.
        \end{theorem*}
    \end{mdframed}

    The theorem itself does not exclude multiple numbers \(s\), but the following proof will rule that out.

    \begin{theorem}
        The number \(s\) is unique.
    \end{theorem}

    \begin{proof}
        Let's say two numbers \(s \neq t\) are included in every interval. Without loss of generality, say \(s < t\). Then every interval is of length \(\geq t - s\), which contradicts (I.2). \lightning
    \end{proof}

    \subsection{Existence of roots}

    With the introduction of interval nestings, we can now prove a theorem that directly constructs roots. Some prerequisites first, which we will use.

    \begin{theorem}[Archimedes Axiom]
        For every \(x \in \mathbb{R}\), there is a \(n \in \mathbb{N}\), so that
        \[
            n > x
        \]
    \end{theorem}

    The statement of this axiom is clear. It goes along with two more axioms in the ordering of the real numbers, which is not mentioned here.

    \begin{lemma}
        %\label{lemma1}
        Let \(a > b\). Then
        \[
            \frac{1}{a} < \frac{1}{b}
        \]
        for the reciprocals.
    \end{lemma}

    This is a well known theorem. We will not prove it here, but it is important for the following proofs.

    \begin{lemma}[Bernoulli's Inequality]
        Let \(x \in \mathbb{R}\) with \(x \neq 0, x > -1\) and \(n \in \mathbb{N}, n \geq 2\). Then the following holds
        \[
            (1+x)^n > 1 + nx
        \]
    \end{lemma}

    \begin{proof}[Proof by induction after \(n\).] \phantom{}\\
        \emph{Base case})
            Let \(n = 2\). Then
            \[
                (1+x)^2 \geq 1 + 2x \Leftrightarrow x^2+2x+1 \geq 1 + 2x
            \]
            which holds for every \(x\) specified.
        
        \(n \leadsto n + 1\))
            Consider the following
            \begin{align*}
                (1+x)^{n+1} &= (1+x)^n \cdot (1+x)\\
                            &\geq (1+nx) \cdot (1+x) & \text{True for} < n + 1\\
                            &\geq 1+(n+1)x
            \end{align*}
            By the principle of induction, the statement is true.
    \end{proof}

    This lemma will be useful for the next one, which we will use later to prove (I.2) for constructed interval nestings.

    \begin{lemma}
        Let \(0 < q < 1\). For every \(\varepsilon > 0\), there exists an \(n \in \mathbb{N}\) so that the following holds
        \[
            q^n < \varepsilon
        \]
    \end{lemma}

    \begin{proof}
        Write \(q' \coloneqq q^{-1}\) and \(\varepsilon' \coloneqq \varepsilon^{-1}\). Since \(q' > 1\) due to the ordering in \(\mathbb{R}\), we can write \(q' = 1 + x\) for \(x \coloneqq q' - 1\). Therefore we can use Bernoulli's Inequality above:
        \[
            (1+x)^n > 1 + nx \text{ for one } n \geq 2
        \]
        Which \(n\) do we choose? We use the Archimedes Axiom. There is a natural \(n \geq 2\) such that \(n > \frac{\varepsilon'}{x}\) and therefore
        \[
            (q')^n > 1 + nx > nx > \varepsilon'
        \]
        Now we have \((q')^n > \varepsilon'\). If we take the reciprocals, then according to the lemma above, we get the theorem.
        \[
            q^n < \varepsilon
        \]
    \end{proof}

    Now for the main part.

    \begin{theorem}
        For every real number \(x > 0\) and every \(k \in \mathbb{N}\), there is one and only one real number \(y > 0\) with \(y^k = x\). We call it the \(k\)th root of \(x\), in symbols \(y = x^\frac{1}{x}, y = \sqrt[k]{x}\).
    \end{theorem}

    \begin{proof}
    We consider the case \(x > 1\), since for \(x < 1\) we can make the transition \(x' \coloneqq 1/x\), which complies with the ordering of \(\mathbb{R}\). And for \(x = 1\) it holds that \(y = 1\).\\

    First, we construct an interval nesting \((I_n)\) in the \(\mathbb{R}_+\).
    
    We begin with the properties our nesting should hold. For every interval \(I_n = [a_n; b_n], n \in \mathbb{N}\), the following should hold

    \begin{itemize}[label={}]
        \item[(\(1_n\))] \(a_n^k \leq x \leq b_n^k\)
        
        \item[(\(2_n\))] \(|\intv_n| = \left(\frac{1}{2}\right)^{n-1}\cdot|\intv_1|\)
    \end{itemize}

    So every interval interval \(\intv_{n + 1}\) is half the length of the previous interval \(\intv_n\).\\

    Now for the construction we use induction. We declare the first interval and then, by the properties of the natural numbers, declare the next ones.

    Let \(\intv_1 \coloneqq [1; x]\). We verify that the properties (\(1_1\)) and (\(2_1\)) hold.
    
    Now let \(n\) be arbitrary, but fixed, with \(\intv_n = [a_n; b_n]\) holding the properties (\(1_n\)) and (\(2_n\)).
    
    We construct the next interval \(n + 1\) by cutting the \(n\)th one in half. Let \(m \coloneqq \frac{b_n - a_n}{2}\) be the center of the interval. We then define
    \[
        \intv_{n+1} \coloneqq \begin{cases}
            [a_n; m] & m \geq x\\
            [m; b_n] & m < x
        \end{cases}
    \]

    Due to our construction, (\(1_{n+1}\)) and (\(2_{n+1}\)) both hold, since the newly constructed interval is in the bounds of \(\intv_n\) and therefore included. And because the new interval is exactly half the previous one.\\

    We must now still prove that both (I.1) and (I.2) hold.

    \begin{itemize}[label={}]
        \item[(I.1)]
        See above. % TODO: Add refs

        \item[(I.2)]
        With \((2_n)\) consider
        \[
            \left(\frac{1}{2}\right)^{n-1=n'} < \varepsilon' = \varepsilon \cdot |\intv_1|^{-1}
        \]
        With the lemma we have proven above, there exists such an integer \(n'\) and therefore
        \[
            \left(\frac{1}{2}\right)^{n-1} \cdot |\intv_1| < \varepsilon 
        \]
        which satisfies the property.
    \end{itemize}

    Let \(y\) be the number that is included  in all intervals \((\intv_n)\). We will now show that \(y^k = x\).\\

    For that, we first prove that the intervals \(([a_n^k; b_n^k])\) also form an interval nesting.\\
    
    
    Since the first intervals all include \(y\), these ones include \(y^k\), which is in the bounds of the \(a_n^k, b_n^k\). We also know, that all these intervals include the number \(x\). Due to the uniqueness of the contained number, we derive that \(x = y^k\) and therefore \(y = \sqrt[k]{x}\).

    \end{proof}

    \section{Implementation}

    \subsection{Algorithm design}

    \subsection{Complexity considerations}

    \printbibliography
\end{document}
    