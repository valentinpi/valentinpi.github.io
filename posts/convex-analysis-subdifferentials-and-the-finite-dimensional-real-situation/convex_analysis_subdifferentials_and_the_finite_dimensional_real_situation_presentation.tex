\documentclass[10pt, hyperref={hidelinks}]{beamer}

%\usepackage{algorithm}
%\usepackage[noend]{algpseudocode}
\usepackage{amsfonts}
\usepackage{amsmath}
\usepackage{amssymb}
\usepackage{amsthm}
\usepackage[ngerman]{babel}
\usepackage[backend=biber, citestyle=numeric-comp, bibstyle=ieee]{biblatex}
%\usepackage{changepage}
\usepackage{enumitem}
%\usepackage{fancyhdr}
%\usepackage{fontspec}
%\usepackage{fullpage}
\usepackage{mathtools}
\usepackage{physics}
\usepackage{thmtools}
\usepackage{tikz}
\usepackage{tikz-3dplot}
\usetikzlibrary{angles, cd, quantikz, quotes, patterns}
%\usepackage{titlesec}
\usepackage{wasysym}

% Oh, templates can customize different aspects of a presentation?!
\setbeamertemplate{theorems}[numbered]

%\usepackage{tikz-cd}

%\usepackage{bookmark}
\usepackage[nameinlink]{cleveref}

\usecolortheme{dove}

%\titleformat{\section}[runin]{\normalsize\bfseries}{\thesection}{1em}{}[]
%\titleformat{\subsection}[runin]{\normalsize\bfseries}{\thesubsection}{1em}{}[]
%\titleformat{\subsubsection}[runin]{\normalsize\bfseries}{\thesubsubsection}{1em}{}[]

\addbibresource{convex_analysis_subdifferentials_and_the_finite_dimensional_real_situation_presentation.bib}

%\theoremstyle{definition}
%\newtheorem{theorem}{Theorem}
%\newtheorem{definition}{Definition}
%\theoremstyle{remark}
%\newtheorem{problem}[theorem]{Problem}
%\newtheorem{lemma}[theorem]{Lemma}
%\newtheorem{remark}[theorem]{Remark}
%\newtheorem{observation}[theorem]{Observation}
%\newtheorem{example}[theorem]{Example}
%\newtheorem{corollary}[theorem]{Corollary}

\renewcommand{\qedsymbol}{\(\blacksquare\)}

\setlength{\parindent}{0pt}

\DeclareMathOperator{\controrot}{CR}
\DeclareMathOperator{\expectation}{E}
\DeclareMathOperator{\gf}{GF}
\DeclareMathOperator{\qft}{QFT}
\DeclareMathOperator{\rk}{rk}
\DeclareMathOperator{\defect}{def}
\DeclareMathOperator{\swapgate}{SWAP}
\DeclareMathOperator{\che}{CHE}
\DeclareMathOperator{\poly}{poly}
\DeclareMathOperator{\Span}{Span}
\DeclareMathOperator{\diag}{diag}

\newcommand{\djk}{\delta_{j, k}}
\newcommand{\tlk}{\tilde{\lambda_k}}

\newcommand{\evalat}[2]{\left.{#1}\middle|\right._{#2}}

% SOURCE: https://tex.stackexchange.com/questions/296151/double-head-and-hook-arrow
\newcommand{\hookdoubleheadrightarrow}{%
  \hookrightarrow\mathrel{\mspace{-15mu}}\rightarrow
}

\newcommand{\draftcomment}[2]{\textcolor{#1}{#2}}

\title{Presentation Notes on: Further Properties of Subdifferentials and The Situation in \(\mathbb{R}^n\)}

\author{valentinpi\\\phantom{}\\\textsc{Seminar on Convex Analysis}\\Freie Universität Berlin\\Summer Term 2023}

\date{July 25, 2023}

\begin{document}
    \frame{\titlepage}

    \begin{frame}
        \frametitle{Recap}

        \pause
    
        Let \(X, Y\) be locally convex topological vector spaces, \(n \in \mathbb{N}_{\geq 1}\) and \(f\colon X \to \overline{\mathbb{R}}\).

        \pause
        Recall:
        \begin{itemize}
            \item Effective domain, epigraph of a function.
            \pause
            \item Convexity of sets and functions.
            \pause
            \item \(f\) is proper, if \(-\infty < f \neq \infty\).
            \pause
            \item \(x^* \in X^*\) is a subgradient of \(f\) at \(x \in X\), if \(f(z-x) \geq f(x) + \langle x^*, z-x \rangle \, \forall \, z \in X\).
            \pause
            \item The set of all subgradients is the subdifferential \(\partial f(x)\).
        \end{itemize}
        \pause
        So far: Definition, characterization, Moreau-Rockefellar.
        
        \pause
        Today: More properties, some for \(X = \mathbb{R}^n\).
    \end{frame}

    \begin{frame}
        \frametitle{Recap: Important General Characterizations}

        \begin{theorem}[{\cite[pp. 170-171]{IoffeTihomirov}}] \label{proper_convex_continuity_boundedness_equivalence}
            Let \(f\) be proper and convex. Then the following statements are equivalent.
            \begin{enumerate}[label=(\roman*), wide]
                \item \(f\) is bounded on a neighborhood of \(x \in X\).
                \item \(f\) is continuous at \(x\).
            \end{enumerate}
        \end{theorem}
    \end{frame}

    \begin{frame}
        \frametitle{Recap: Important General Characterizations}

        \begin{theorem}[{\cite[pp. 193-199]{IoffeTihomirov}}] \label{convex_subdifferential_characterizations}
            Let \(f\) be convex. Then the following statements are equivalent.
            \begin{enumerate}[label=(\roman*), wide]
                \item \(x^* \in \partial f(x)\).
                \item \(f(z) \geq f(x) + \langle x^*, z-x \rangle \, \forall \, z \in X\).
                \item \(f(x)+f^*(x^*) = \langle x^*, x \rangle\)
                \item \label{convex_subdifferential_characterizations_4} \(f'(x, y) = \sup_{x^* \in \partial f(x)} \langle x^*, y \rangle \geq \langle x^*, y \rangle \, \forall \, y \in X\), i.e. \(x^* \in \partial f'(x; 0) = \partial f(x) = \text{dom}(f'(x; \cdot)^*)\), if \(f\) is additionally proper.
            \end{enumerate}
        \end{theorem}
    \end{frame}

    \begin{frame}
        \frametitle{Recap: Important General Characterizations}
    
        \begin{theorem}[{Infimum Form \cite[pp. 194-195]{IoffeTihomirov}}]
            Let \(f\) be proper and convex. Then \(f'(x; \cdot)\) is well defined for any \(x \in \text{dom}(f)\), convex, proper and for any \(y \in X\)
            \begin{align*}
                f'(x; y) = \inf_{\lambda \in \mathbb{R}_{> 0}} \frac{f(x+\lambda y)-f(x)}{\lambda}
            \end{align*}
        \end{theorem}
    \end{frame}

    \begin{frame}
        \frametitle{A Chain Rule}

        \pause
        \begin{theorem}[{\cite[p. 201]{IoffeTihomirov}}]
            Let \(\Lambda \in L(X, Y)\) and \(f\colon Y \to \overline{\mathbb{R}}\) be a function.
            \pause
            \begin{enumerate}[label=(\roman*), wide]
                \item \label{compat_with_operator_theorem_1} For any \(x \in X\), we have
                \begin{align*}
                    \partial (f\Lambda)(x) \supseteq \Lambda^*\partial f(\Lambda x)
                \end{align*}
                \pause
                \item \label{compat_with_operator_theorem_2} If \(f\) is convex and proper on \(X\), as well as continuous at a point in \(\text{Im}(\Lambda)\), then for any \(x \in X\)
                \begin{align*}
                    \partial (f\Lambda)(x) = \Lambda^*\partial f(\Lambda x) \label{compat_with_operator_theorem_eq_2}
                \end{align*}
            \end{enumerate}
        \end{theorem}
        \pause
        \begin{figure}[!hbtp]
            \begin{tikzpicture}[>=stealth, semithick]
                \node (0) at (0, 0) {\(X\)};
                \node[below=1cm of 0] (1) {\(Y\)};
                \node[right=1cm of 1] (2) {\(\mathcal{P}(Y^*)\)};
                \node[right=1cm of 2] (3) {\(\mathcal{P}(X^*)\)};
                \draw[->] (0) -- (1) node[left, pos=0.5] {\(\Lambda\)};
                \draw[->] (1) -- (2) node[below, pos=0.5] {\(\partial f\)};
                \draw[->] (2) -- (3) node[below, pos=0.5] {\(\Lambda^*\)};
                \draw[->] (0) -- (3) node[above right, pos=0.5] {\(\partial (f\Lambda)\)};
            \end{tikzpicture}
        \end{figure}
    \end{frame}

    \begin{frame}
        \frametitle{A Chain Rule: Helper Theorems}
        \pause
        \begin{theorem}[{\cite[pp. 195-196]{IoffeTihomirov}}] \label{point_directional_derivative_continuity_theorem}
            Let \(f\colon X \to \overline{\mathbb{R}}\) be a proper, convex function, which is continuous on \(\emptyset \neq U \subseteq X\) and \(x \in X\) be fixed.
            \begin{enumerate}[label=(\roman*), wide]
                \item \label{point_directional_derivative_continuity_theorem_1} If \(|f'(x; \bar{x})| < \infty\) for \(\bar{x} \in X\) with \(x+\bar{x} \in U\), then \(f'(x; \cdot)\) is continuous on \(K_{U \setminus \{x\}} \setminus \{0\}\) or \(K_{U \setminus \{x\}}\).
                \item \label{point_directional_derivative_continuity_theorem_2} If \(f\) is continuous at \(x\), then \(f'(x; \cdot)\) is finite and continuous on \(X\).
            \end{enumerate}
        \end{theorem}
        \pause
        \begin{theorem}[{\cite[p. 179, p. 183]{IoffeTihomirov}}] \label{adjoint_calculation_rule}
            Let \(\Lambda \in L(X, Y)\) and \(f\colon Y \to \overline{\mathbb{R}}\) be a convex function, continuous at a point in \(\Im(\Lambda)\). Then \((f\Lambda)^* = \Lambda^* f^*\) and for each \(x^* \in \text{dom}((f\Lambda)^*)\), there is a \(y^* \in Y^*\) with \(x^* = \Lambda^*y^*\) and \((f\Lambda)^*(x^*) = f^*(y^*)\).
        \end{theorem}
    \end{frame}

    \begin{frame}
        \frametitle{The Supremum Function of Convex Functions is Convex}

        \pause
    
        \begin{lemma} \label{supremum_convexity_lemma}
            Let \(\{f_s\colon X \to \overline{\mathbb{R}}\}_{s \in S}\), with \(S\) an index set, be a family of convex functions and \(\hat{f}\colon X \to \overline{\mathbb{R}}, x \mapsto \sup_{s \in S} f(x)\). Then \(\hat{f}\) is convex.
        \end{lemma}
    \end{frame}

    \begin{frame}
        \frametitle{Supremum Function Subdifferential}
        
        \pause

        \begin{theorem}[{\cite[pp. 201-204]{IoffeTihomirov}}] \label{supremum_theorem}
            Let \(S\) be a compact topological space and \(f\colon S \times X \to \overline{\mathbb{R}}\), s.t. for any \((s, x) \in S \times X\), \(f|_{\{s\} \times X}\) is convex and proper, and that \(f|_{S \times \{x\}}\) is upper semicontinuous. Let \(f_s \coloneqq f|_{\{s\} \times X}\) and set
            \begin{align*}
                \hat{f}\colon X \to \overline{\mathbb{R}}, x \mapsto \sup_{s \in S} f_s(x) \text{ and } S_0\colon X \to \mathcal{P}(S), x \mapsto \{s \in S \mid f_s(x) = \hat{f}(x)\}.
            \end{align*}
            \pause
            \begin{enumerate}[label=(\roman*), wide]
                \item \label{supremum_theorem_1} For any \(x \in X\)
                \begin{align*}
                    \overline{\text{conv}}\left(\bigcup_{s \in S_0(x)} \partial f_s(x)\right) \subseteq \partial \hat{f}(x)
                \end{align*}
                \pause
                \item \label{supremum_theorem_2} If for all \(s \in S\), \(f_s\) is continuous at a point \(x_0 \in X\), then
                \begin{align*}
                    \overline{\text{conv}}\left(\bigcup_{s \in S_0(x_0)} \partial f_s(x_0)\right) = \partial \hat{f}(x_0)
                \end{align*}
            \end{enumerate}
        \end{theorem}
    \end{frame}

    \begin{frame}
        \frametitle{Supremum Function Subdifferential: Helper Separation Theorem}

        \pause
    
        \begin{theorem}[{\cite[pp. 164-165]{IoffeTihomirov}}] \label{separation_theorem}
            Let \(A \subseteq X\) be closed, convex and let \(x \notin A\). Then there exists a functional \(x^* \in X^*\), s.t. \(\langle x^*, y\rangle \leq \langle x^*, x \rangle - \varepsilon\) for a fixed \(\varepsilon \in \mathbb{R}_{> 0}\) and any \(y \in A\), strongly separating \(A\) and \(\{x\}\).
        \end{theorem}
    \end{frame}

    \begin{frame}
        \frametitle{Subdifferentials in \(\mathbb{R}^n\): Existence}

        \pause
    
        \begin{theorem}[{\cite[p. 204]{IoffeTihomirov}}]
            Let \(f\colon \mathbb{R}^n \to \overline{\mathbb{R}}\) be proper and convex. Then \(f\) is subdifferentiable in \(\text{ri}(\text{dom}(f))\).
        \end{theorem}

        \pause

        \phantom{}

        One helper theorem is needed.

        \phantom{}

        \pause

        \begin{lemma}[{\cite[p. 188]{IoffeTihomirov}}] \label{proper_convex_continuity_lemma}
            Let \(f\colon \mathbb{R}^n \to \overline{\mathbb{R}}\) be convex and proper.
            \begin{enumerate}[label=(\roman*), wide]
                \item \label{proper_convex_continuity_lemma_1} \(f\) is continuous with respect to \(\text{aff}({\text{dom}(f)})\) on \(\text{ri}(\text{dom}(f))\).
                \item \label{proper_convex_continuity_lemma_2} \(f^*\) is proper.
            \end{enumerate}
        \end{lemma}
    \end{frame}

    \begin{frame}
        \frametitle{Subdifferentials in \(\mathbb{R}^n\): Representation}

        \pause

        \begin{theorem}[{\cite[pp. 204-205]{IoffeTihomirov}}] \label{finite_dim_representation_theorem}
            Let \(X, S, f, f_s, \hat{f}, S_0, x_0\) for \(X = \mathbb{R}^n\) and any \(s \in S\) be defined as in the setting of \Cref{supremum_theorem} \ref{supremum_theorem_2}. Then every \(y \in \partial \hat{f}(x_0)\) can be represented as a convex combination of form
            \begin{align*}
                y = \sum_{i=1}^r \alpha_i y_i
            \end{align*}
            with \(r \in \mathbb{N}\), \(1 \leq r \leq n+1\), \(\sum_{i=1}^r \alpha_i = 1\) and \(s_i \in S_0(x_0)\), \((\alpha_i, y_i) \in \mathbb{R}_{>0} \times \partial f_{s_i}(x_0)\) for any \(i \in \mathbb{N}\), \(1 \leq i \leq r\).
        \end{theorem}
    \end{frame}

    \begin{frame}
        \frametitle{Subdifferentials in \(\mathbb{R}^n\): Representation - Helper Theorems}

        \pause
        
        \begin{lemma}[{\cite[pp. 185-186]{IoffeTihomirov}}] \label{finite_dim_caratheodory_corollary}
            Let \(A \subseteq \mathbb{R}^n\) be bounded and closed. Then \(\text{conv}(A) = \overline{\text{conv}}(A)\).
        \end{lemma}

        \pause
        
        \begin{lemma}[{\cite[p. 199]{IoffeTihomirov}}] \label{subdifferential_boundedness}
            For a proper convex function \(f\colon X \to \overline{\mathbb{R}}\), which is continuous at a point \(x_0\), \(\partial f(x_0)\) is non-empty, weakly\({}^*\) bounded and weakly\({}^*\) compact.
        \end{lemma}

        \pause
        
        \begin{lemma} \label{s_zero_s_zero_compact_lemma}
            Let \(X, S, f, f_s, \hat{f}, S_0, x_0\) be defined as in the setting of \Cref{supremum_theorem} \ref{supremum_theorem_2}. Then \(S_0(x_0)\) is compact.
        \end{lemma}
    \end{frame}

    \begin{frame}
        \frametitle{References}

        \printbibliography{}
    \end{frame}
\end{document}
