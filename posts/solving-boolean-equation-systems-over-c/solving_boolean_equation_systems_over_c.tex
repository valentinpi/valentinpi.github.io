\documentclass[10pt]{amsart}

\usepackage{algorithm}
\usepackage[noend]{algpseudocode}
\usepackage{amsfonts}
\usepackage{amsmath}
\usepackage{amssymb}
\usepackage{amsthm}
\usepackage[backend=biber, citestyle=numeric-comp, bibstyle=ieee]{biblatex}
\usepackage{changepage}
\usepackage{enumitem}
\usepackage{fancyhdr}
\usepackage{fontspec}
\usepackage{fullpage}
\usepackage[hidelinks]{hyperref}
\usepackage{mathtools}
\usepackage{physics}
\usepackage{thmtools}
\usepackage{tikz}
\usepackage{tikz-3dplot}
\usetikzlibrary{angles, cd, quantikz, quotes, patterns}
\usepackage{titlesec}
\usepackage{wasysym}

\usepackage{tikz-cd}

\usepackage{bookmark}
\usepackage[nameinlink]{cleveref}

\titleformat{\section}[runin]{\normalsize\bfseries}{\thesection}{1em}{}[]
\titleformat{\subsection}[runin]{\normalsize\bfseries}{\thesubsection}{1em}{}[]
\titleformat{\subsubsection}[runin]{\normalsize\bfseries}{\thesubsubsection}{1em}{}[]

\addbibresource{solving_boolean_equation_systems_over_c.bib}

\theoremstyle{definition}
\newtheorem{theorem}{Theorem}
\newtheorem{definition}{Definition}
\theoremstyle{remark}
\newtheorem{problem}[theorem]{Problem}
\newtheorem{lemma}[theorem]{Lemma}
\newtheorem{remark}[theorem]{Remark}
\newtheorem{observation}[theorem]{Observation}
\newtheorem{example}[theorem]{Example}
\newtheorem{corollary}[theorem]{Corollary}

\renewcommand{\qedsymbol}{\(\blacksquare\)}

\setlength{\parindent}{0pt}

\DeclareMathOperator{\controrot}{CR}
\DeclareMathOperator{\expectation}{E}
\DeclareMathOperator{\gf}{GF}
\DeclareMathOperator{\qft}{QFT}
\DeclareMathOperator{\rk}{rk}
\DeclareMathOperator{\defect}{def}
\DeclareMathOperator{\swapgate}{SWAP}
\DeclareMathOperator{\che}{CHE}
\DeclareMathOperator{\poly}{poly}
\DeclareMathOperator{\Span}{Span}
\DeclareMathOperator{\diag}{diag}

\newcommand{\djk}{\delta_{j, k}}
\newcommand{\tlk}{\tilde{\lambda_k}}

\newcommand{\evalat}[2]{\left.{#1}\middle|\right._{#2}}

% SOURCE: https://tex.stackexchange.com/questions/296151/double-head-and-hook-arrow
\newcommand{\hookdoubleheadrightarrow}{%
    \hookrightarrow\mathrel{\mspace{-15mu}}\rightarrow
}

\newcommand{\draftcomment}[2]{\textcolor{#1}{#2}}

\begin{document}
    \section*{Solving Boolean Equation Systems over \(\mathbb{C}\)} \hfill \hfill valentinpi, Last Change: \today{}

    \phantom{}

    The following notes are based on \cite{Chen_2021}. Let \(X = \{x_1, ..., x_n\}\) be a set of variables and \(\mathcal{F} = \{f_1, ..., f_r\} \subseteq \mathbb{F}_2[X]\) be a set of Boolean multivariate polynomials with \(n, r \in \mathbb{N}_{\geq 1}\). We assume all polynomials have no redundant terms and omit the explicit indexing in the following. Denote by \(\mathbb{V}_F(\mathcal{F}) = \bigcap_{i=1}^r f_i|_F^F{}^{-1}(0)\) the \emph{variety} of \(\mathcal{F}\) over a field \(F\), where \(f_i|_F^F\colon F \to F\) is the polynomial function over \(F\) associated with \(f_i\) and \(t_i \in \mathbb{N}_{\geq 1}\) the number of (monomial) summands in \(f_i\) for each \(i\).

    \phantom{}

    The overall goal is to solve \(\mathcal{F}\) over \(\mathbb{F}_2\) by solving it over \(\mathbb{C}\). In general, \(\mathbb{V}_{\mathbb{F}_2}(\mathcal{F}) \neq \mathbb{V}_{\mathbb{C}}(\mathcal{F})\). For instance, for the system \(\mathcal{F} = \{x_1x_2\}\), we have \(\mathbb{V}_{\mathbb{F}_2}(\mathcal{F}) = \{(0, 0), (0, 1), (1, 0)\} \neq \mathbb{C} \times \{0\} \cup \{0\} \times \mathbb{C} = \mathbb{V}_{\mathbb{C}}(\mathcal{F})\).

    \phantom{}

    The first step is to restrict the variables in the solution to \(\mathbb{F}_2\). Thus, consider the system \(\mathcal{F} \cup \{x_i^2-x_i\}_i\). Over \(\mathbb{C}\), the added polynomials suffice \(x_i^2-x_i = x_i(x_i-1)\) for each \(i\), so the solution has to restrict the variables to \(0\) or \(1\).

    \phantom{}

    However, this is not sufficient. Consider the system \(\mathcal{F} = x_1^2+x_1x_2\). We have \(\mathbb{V}_{\mathbb{F}_2}(\mathcal{F}) = \{(0, 0), (0, 1), (1, 1)\}\), but \(\mathbb{V}_{\mathbb{C}}(\mathcal{F} \cup \{x_1^2-x_1, x_2^2-x_2\}) = \{(0, 0), (0, 1)\}\). So we adjust the general system by introducing
    \begin{align}
        C(f_i) \coloneqq \prod_{k=f_i|_{\mathbb{F}_2}^{\mathbb{F}_2}(0)}^{\lfloor t_i/2 \rfloor} (f_i-2k)
    \end{align}
    for each \(i\) and letting
    \begin{align}
        C(\mathcal{F}) \coloneqq \{C(f_i)\}_i.
    \end{align}
    \begin{lemma}[{\cite[p. 23]{Chen_2021}}]
        We have \(\mathbb{V}_{\mathbb{F}_2}(\mathcal{F}) = \mathbb{V}_{\mathbb{C}}(C(\mathcal{F}))\).
    \end{lemma}
    \begin{proof}
        \((\subseteq)\) Let \(f_i = \sum_{k=1}^{t_i} m_{ik}\) with monomials \(m_{ik} \in \mathbb{F}_2[X]\), \(i\) be arbitrarily chosen, and \(a \in \mathbb{V}_{\mathbb{F}_2}(\mathcal{F})\). Thus \(f_i|_{\mathbb{C}}^{\mathbb{C}}(a) \in [f_i|_{\mathbb{F}_2}^{\mathbb{F}_2}(0), t_i]_{\mathbb{N}} \cap 2\mathbb{N}\), implying \(C(f_i)(a) = 0\).

        \((\supseteq)\) On the other hand, let \(i\) be arbitrary and \(a \in \mathbb{V}_{\mathbb{C}}(C(\mathcal{F}))\). Then \(f_i|_{\mathbb{C}}^{\mathbb{C}}(a) = 2k\) for a \(k \in [f_i|_{\mathbb{F}_2}^{\mathbb{F}_2}(0), \lfloor t_i/2 \rfloor]_{\mathbb{N}}\), so \(a \in \mathbb{V}_{\mathbb{F}_2}(\mathcal{F})\).
    \end{proof}

    \printbibliography{}
\end{document}
