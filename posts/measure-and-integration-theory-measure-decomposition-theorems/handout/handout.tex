\documentclass[10pt, leqno]{amsart}

\usepackage{algorithm}
\usepackage[noend]{algpseudocode}
\usepackage{amsfonts}
\usepackage{amsmath}
\usepackage{amssymb}
\usepackage{amsthm}
\usepackage[backend=biber, citestyle=numeric-comp, bibstyle=ieee]{biblatex}
\usepackage{changepage}
\usepackage{enumitem}
\usepackage{fancyhdr}
\usepackage{fontspec}
\usepackage{fullpage}
\usepackage[hidelinks]{hyperref}
\usepackage{marvosym}
\usepackage{mathtools}
\usepackage[]{mdframed}
\usepackage{physics}
\usepackage{tabularx}
\usepackage[skins]{tcolorbox}
\usepackage{thmtools}

\usepackage{tikz-3dplot}
\usetikzlibrary{angles, calc, cd, quantikz, quotes, patterns}
\usepackage{titlesec}
\usepackage{wasysym}

\usepackage{tikz-cd}

\usepackage{bookmark}
\usepackage[nameinlink]{cleveref}

\titleformat{\section}[runin]{\normalsize\bfseries}{\thesection}{1em}{}[]
\titleformat{\subsection}[runin]{\normalsize\bfseries}{\thesubsection}{1em}{}[]
\titleformat{\subsubsection}[runin]{\normalsize\bfseries}{\thesubsubsection}{1em}{}[]

\addbibresource{handout.bib}

\theoremstyle{definition}
\newtheorem{theorem}{Theorem}[section]
\newtheorem{conjecture}{Conjecture}[section]
\newtheorem{definition}{Definition}[section]
\theoremstyle{remark}
\newtheorem{problem}[theorem]{Problem}
\newtheorem{lemma}[theorem]{Lemma}
\newtheorem{remark}[theorem]{Remark}
\newtheorem{observation}[theorem]{Observation}
\newtheorem{example}[theorem]{Example}
\newtheorem{corollary}[theorem]{Corollary}

\renewcommand{\qedsymbol}{\(\blacksquare\)}

\setlength{\parindent}{0pt}

\renewcommand{\theequation}{\arabic{equation}}
\renewcommand{\thetheorem}{\arabic{theorem}}

\DeclareMathOperator{\controrot}{CR}
\DeclareMathOperator{\expectation}{E}
\DeclareMathOperator{\gf}{GF}
\DeclareMathOperator{\qft}{QFT}
\DeclareMathOperator{\rk}{rk}
\DeclareMathOperator{\defect}{def}
\DeclareMathOperator{\swapgate}{SWAP}
\DeclareMathOperator{\che}{CHE}
\DeclareMathOperator{\poly}{poly}
\DeclareMathOperator{\Span}{Span}
\DeclareMathOperator{\diag}{diag}

\newcommand{\djk}{\delta_{j, k}}
\newcommand{\tlk}{\tilde{\lambda_k}}

\newcommand{\evalat}[2]{\left.{#1}\middle|\right._{#2}}

% SOURCE: https://tex.stackexchange.com/questions/296151/double-head-and-hook-arrow
\newcommand{\hookdoubleheadrightarrow}{%
  \hookrightarrow\mathrel{\mspace{-15mu}}\rightarrow
}

\newtcolorbox{edgebox}{enhanced, colback=white, frame code={%
\draw[very thin] (frame.north west) -- ($(frame.north west) + ( 0.5, 0)$);
\draw[very thin] (frame.north west) -- ($(frame.north west) + (0, -0.5)$);
\draw[very thin] (frame.south east) -- ($(frame.south east) + (-0.5, 0)$);
\draw[very thin] (frame.south east) -- ($(frame.south east) + (0,  0.5)$);
}}

\AtEveryBibitem{%
  \clearfield{journaltitle}%
  \clearfield{date}%
  \clearfield{volume}%
  \clearfield{pages}%
  \clearfield{publisher}%
  \clearfield{number}%
  \clearfield{journaltitle}%
}

\newcommand{\commentcmd}[1]{}

\newcommand{\draftcommenttodo}{\textcolor{red}{ TODO }}
\newcommand{\draftcommentdone}{\textcolor{green}{ DONE }}

\newcolumntype{L}[1]{>{\raggedright\arraybackslash}p{#1}}
\newcolumntype{C}[1]{>{\centering\arraybackslash}p{#1}}

\begin{document}
    \pagenumbering{gobble}

    \begin{mdframed}
        \textsc{Seminar on Measure and Integration Theory} \hfill valentinpi\\
        Freie Universität Berlin \hfill February 18, 2025\\
        Winter Term 2024-25
    \end{mdframed}

    \section*{Presentation Handout on: Measure Decomposition Theorems} \phantom{}
    
    For this presentation, we used mainly \cite[pp. 71-76, pp. 113-118]{Fonseca}, but also \cite{Fonseca} as a whole and \cite{Axler,Elstrodt}.

    \section{Decomposition Theorems for Measures} \phantom{}
    
    Throughout this document, assume that \((X, \mathfrak{M})\) is a measurable space.

    \begin{definition} \label{def:basics_on_measure_relations}
        Let \(\mu, \nu\colon \mathfrak{M} \to [0, \infty]\) be two measures. We define the following three set functions:
        \begin{align}
            \nu_{ac}&\colon \mathfrak{M} \to [0, \infty], E \mapsto \max\left\{\int_E u \, d\mu \, \middle\vert \, u\colon X \to [0, \infty] \text{ measurable} \land \int_{E'} u \, d\mu \leq \nu(E') \, \forall \, \mathfrak{M} \ni E' \subseteq E\right\}\\
            \nu_s&\colon \mathfrak{M} \to [0, \infty], E \mapsto \max\{\nu(E') \mid \mathfrak{M} \ni E' \subseteq E, \mu(E') = 0\} \\
            \nu_d&\colon \mathfrak{M} \to [0, \infty], E \mapsto \max\{\nu(E') \mid \mathfrak{M} \ni E' \subseteq E, \mu(E'') = \infty \, \forall \, \mathfrak{M} \in E'' \subseteq E', \nu(E'') > 0\}
        \end{align}
    \end{definition}

    \begin{theorem}[Lebesgue Decomposition Theorem] \label{thm:lebesgue_decomposition_theorem}
        Let \(\mu, \nu\colon \mathfrak{M} \to [0, \infty]\) be two measures and \(\mu\) \(\sigma\)-finite.
        \begin{enumerate}[label=(\roman*), wide]
            \item \label{thm:lebesgue_decomposition_theorem_1} Then
            \begin{align}
                \nu = \nu_{ac} + \nu_s.
            \end{align}
            \item \label{thm:lebesgue_decomposition_theorem_2} If \(\nu\) is \(\sigma\)-finite, then \(\nu_s \perp \mu\) and the decomposition is unique.
        \end{enumerate}
    \end{theorem}

    \begin{theorem}[De Giorgis Theorem]
        Let \(\mu, \nu\colon \mathfrak{M} \to [0, \infty]\) be two measures. Then
        \begin{align}
            \nu = \nu_{ac} + \nu_s + \nu_d.
        \end{align}
    \end{theorem}

    \section{Decomposition Theorems for Signed Measures} \phantom{}

    Let \(\lambda\colon \mathfrak{M} \to [-\infty, \infty]\) be a signed measure.

    \begin{lemma} \label{lem:existence_of_positive_subset}
        Let \(E \in \mathfrak{M}\) with \(\lambda(E) \in (0, \infty)\). Then there exists a positive \(\mathfrak{M} \ni F \subseteq E\).
    \end{lemma}

    \begin{theorem}[Hahn Decomposition Theorem] \label{thm:hahn_decomposition}
        Any measurable space \((X, \mathfrak{M})\) can be decomposed as \(X = X^+ \cup X^-\), where \(X^+ \subseteq X\) is positive and \(X^- \subseteq X\) is negative.
    \end{theorem}

    \begin{theorem}[Jordan Decomposition Theorem] \label{thm:jordan_decomposition}
        There exists a unique pair \((\lambda^+, \lambda^-)\) of measures with \(\lambda^+ \perp \lambda^-\), one being finite and \(\lambda = \lambda^+ - \lambda^-\).
    \end{theorem}

    \begin{theorem}[Signed Lebesgue Decomposition Theorem]
        Let \(\lambda\colon \mathfrak{M} \to [-\infty, \infty]\) be a signed measure and \(\mu\colon \mathfrak{M} \to [0, \infty]\) be a \(\sigma\)-finite measure.
        \begin{enumerate}[label=(\roman*), wide]
            \item \label{thm:lebesgue_decomposition_thm_1} There are signed measures \(\lambda_{ac}, \lambda_s\colon X \to [-\infty, \infty]\) and a measurable function \(u\colon X \to [-\infty, \infty]\) with
            \begin{align}
                \lambda = \lambda_{ac} + \lambda_s, \lambda_{ac} \ll \mu \text{ and } \lambda_{ac}(\cdot) = \int_{\cdot} u \, d\mu.
            \end{align}
            \item \label{thm:lebesgue_decomposition_thm_2} If \(\lambda\) is \(\sigma\)-finite, then \(\lambda_s \perp \mu\) and the decomposition is unique.
        \end{enumerate}
    \end{theorem}

    \begin{minipage}{\linewidth}
        \centering
        \begin{tabular}{C{4cm}|C{3cm}}
            Decomposition & Summary \\\hline
            Hewitt-Yosida \cite[pp. 8-9]{Fonseca} & \(\mu = \mu_p + \mu_c\)\\
            Atomic \cite[pp. 13-16]{Fonseca} & \(\mu = \mu_1 + \mu_2\)\\
            Lebesgue & \(\nu = \nu_{ac} + \nu_s\)\\
            De Giorgi & \(\nu = \nu_{ac} + \nu_s + \nu_d\)\\
            Hahn & \(X = X^+ \cup X^-\)\\
            Jordan & \(\lambda = \lambda^+ - \lambda^-\)\\
            Signed Lebesgue & \(\lambda = \lambda_{ac} + \lambda_s\)
        \end{tabular}
    \end{minipage}

    \printbibliography{}
\end{document}
