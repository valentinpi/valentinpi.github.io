% Useful links
% List of math symbols: https://oeis.org/wiki/List_of_LaTeX_mathematical_symbols
% Font sizes: http://www.sascha-frank.com/latex-font-size.html

\documentclass[10pt,fleqn]{article}

\usepackage{algorithm}
\usepackage[noend]{algpseudocode}
\usepackage{amsfonts}
\usepackage{amsmath}
\usepackage{amsthm}
%\usepackage[ngerman]{babel}
\usepackage{cancel}
\usepackage{enumitem}
\usepackage{fullpage}
\usepackage{hyperref}
\usepackage{latexsym}
\usepackage{listings}
\usepackage{mathtools}
\usepackage{polynom}
\usepackage{tabularx}
\usepackage{tikz}
\usetikzlibrary{calc}
\usepackage{wasysym}
\usepackage{xcolor}

% https://tex.stackexchange.com/questions/89166/centering-in-tabularx-and-x-columns
% and https://tex.stackexchange.com/questions/257128/how-does-the-newcolumntype-command-work
% Combines c + X in tabularx environments.
% Inserts centering command after the entry in the cell to center it.
\newcolumntype{Y}{>{\centering\arraybackslash}X}

\hypersetup{
    colorlinks   = true,
    urlcolor     = blue,
    linkcolor    = blue,
    citecolor    = red
}

\setlength{\parindent}{0pt}

\theoremstyle{definition}
\newtheorem{definition}{Definition}
\newtheorem{fact}{Fact}
\newtheorem{theorem}{Theorem}
\newtheorem{lemma}[theorem]{Lemma}
\newtheorem{proposition}{Proposition}
\theoremstyle{remark}
\newtheorem{remark}{Remark}

\newcommand{\Task}{x}
\newcommand{\Authors}{valentinpi}
\newcommand{\task}[1]{\item{\bfseries #1}}

\newcommand{\pclass}{\text{P}}
\newcommand{\npclass}{\text{NP}}
\newcommand{\pathprob}{\text{PATH}}
\newcommand{\relprimeprob}{\text{RELPRIME}}
\newcommand{\cflclass}{\text{CFL}}

\newcommand{\lpp}{\left \langle}
\newcommand{\rpp}{\right \rangle}
\newcommand{\enc}[1]{\lpp #1 \rpp}

\DeclareMathOperator{\onot}{\mathcal{O}}
\DeclareMathOperator{\omnot}{\Omega}
\DeclareMathOperator{\thetnot}{\Theta}

\begin{document}
\pagenumbering{arabic}
\title{
    \vspace*{-12ex}
    \phantom{}\\
    \normalsize Task Sheet\\
    \phantom{}\\
    \large The Complexity Class P\\
    \phantom{}\\
    \normalsize Proseminar Theoretische Informatik WiSe 2020-21\\
    \normalsize Institut für Informatik\\
    \normalsize Freie Universität Berlin
}
\author{\normalsize \Authors}
\date{
    \normalsize \today\\
    (neueste Version)\\
    \phantom{}\\
    Due to December 10, 2020 (12 AM)
    \rule{\textwidth}{0.1pt}
}
\maketitle
\begin{enumerate}

    \item{\textbf{TIME Classes} \hfill \emph{(5 pt.)}}
    
    Given the definition of the TIME class, give equivalent definitions for the class SPACE. How can nondeterminism play a role at analyzing complexity?

    \item{\textbf{Problems in P} \hfill \emph{(15 pt.)}}
    
    \begin{enumerate}
        \item Research further problems in P. Is the problem
        \(
            \text{PRIMES} \coloneqq \{ \; \enc{x} \mid x \text{ is prime} \; \}
        \)
        in P?
        \item Prove whether the following problems are in P or SPACE (Check both).
        \[
            \text{H} \coloneqq \{\;\enc{M, w} \mid M \text{ is a TM that will halt on the input } w\;\}
        \]
        \[
            \text{TREES} \coloneqq \{\; \enc{G} \mid G \text{ is an undirected graph and a tree} \;\}
        \]
        \[
            \text{HOARE} \coloneqq \{\; \enc{w} \mid w \text{ is a correctly functioning iterative program} \;\}
        \]
        \[
            \text{BINOMIALCOEFF} \coloneqq \left\{ \; \enc{x} \mid \text{There are natural numbers } n \text{, } k \text{ so that } x = \binom{n}{k} \; \right\}
        \]
        \[
            \text{LINES}_3 \coloneqq \{\; \enc{L_1, ..., L_k} \mid L_1, ..., L_n \subset \mathbb{R}^3 \text{ are lines that pairwise intersect} \;\}
        \]
        \item Update the landscape of languages according to your new knowledge.
    \end{enumerate}

    \item{\textbf{Reducibility in the Context of P} \hfill \emph{(10 pt.)}}
    
    Consider the following problem for any fixed \(n \in \mathbb{N}\):
    \[
        \text{PLANES}_n \coloneqq \{\; \enc{P_1, ..., P_m} \mid P_1, ..., P_m \subset \mathbb{R}^n \text{ are planes that pairwise intersect} \;\}
    \]
    Perform a reduction to another problem that seems suitable, so that you can show that:
    \[
        \text{PLANES}_n \in \pclass
    \]

\end{enumerate}
\end{document}
