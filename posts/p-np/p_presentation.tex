\documentclass[notheorems]{beamer}

\usepackage{algorithm}
\usepackage[noend]{algpseudocode}
\usepackage{amsfonts}
\usepackage{amsmath}
\usepackage{amssymb}
\usepackage{amsthm}
\usepackage[ngerman, english]{babel}
\usepackage{enumitem}
\usepackage{graphicx}
\usepackage{mathtools}
\usepackage{tikz}
\usetikzlibrary{calc}
\usetikzlibrary{arrows}

\usetheme{metropolis}

\setlength{\parindent}{0pt}

% I do not really like the plain style
%\theoremstyle{plain}
\theoremstyle{definition}
\newtheorem{definition}{Definition}
\newtheorem{theorem}{Theorem}
\newtheorem{proposition}{Proposition}
\newtheorem{lemma}[theorem]{Lemma}
\newtheorem{corollary}{Corollary}
\newtheorem{fact}{Fact}
\theoremstyle{remark}
\newtheorem{remark}{Remark}

\newcommand{\pclass}{\text{P}}
\newcommand{\npclass}{\text{NP}}
\newcommand{\pathprob}{\text{PATH}}
\newcommand{\relprimeprob}{\text{RELPRIME}}
\newcommand{\cflclass}{\text{CFL}}

\newcommand{\lpp}{\left \langle}
\newcommand{\rpp}{\right \rangle}
\newcommand{\enc}[1]{\lpp #1 \rpp }

\DeclareMathOperator{\onot}{\mathcal{O}}
\DeclareMathOperator{\omnot}{\Omega}
\DeclareMathOperator{\thetnot}{\Theta}

\renewcommand{\qedsymbol}{\(\blacksquare\)}

\title{The Complexity Class P}
\author{valentinpi}
\institute{\normalsize Proseminar Theoretische Informatik WiSe 2020-21\\
Institut für Informatik\\
Freie Universität Berlin}
\date{8. Dezember 2020{ }(neueste Version)}

\begin{document}
    \selectlanguage{ngerman}
    \frame{\titlepage}
    \selectlanguage{english}

    \begin{frame}
        \begin{center}
            Computability versus Complexity

            \pause

            What kind of efficiency?
        \end{center}
    \end{frame}

    \begin{frame}
        \frametitle{Complexity classes}

        \begin{definition}
            Let \(f\colon \mathbb{N} \rightarrow \mathbb{R}^+\) be a function. The complexity class \emph{TIME} of \(f\) is defined as follows:
            \begin{align*}
                \text{TIME}(f(n)) \coloneqq \{ \; L \mid &\text{There is a deterministic TM that decides } L \text{ in }\\&\onot{(f(n))} \text{ time for an input of size } n \; \}
            \end{align*}
        \end{definition}
    \end{frame}

    \begin{frame}
        \frametitle{The Class P}
    
        \begin{definition}
            \emph{P} is the class of languages decidable in polynomial time on a deterministic single-tape TM.
            \[
                \pclass \coloneqq \bigcup_{k \in \mathbb{N}} \text{TIME}(n^k)
            \]
        \end{definition}
    
    \end{frame}

    \begin{frame}
        \frametitle{The PATH Problem}
        \begin{align*}
            \pathprob \coloneqq \{ \; \enc{G, s, t} \mid &G \text{ is a directed graph and there is a}\\
            &\text{path between nodes } s \text{ and } t \; \}
        \end{align*}
        \begin{figure}[!htbp]
            \centering
            \begin{tikzpicture}[
                semithick
            ]
                \tikzset{every node/.style={fill, circle, minimum size=3pt, inner sep=1pt}}
                \node[label=above left:{\(s\)}] (0) at (0.0, 0.0) {};
                \node (1) at (1.0, 0.5) {};
                \node (2) at (0.5, -0.5) {};
                \node[fill=none, inner sep=1pt] (3) at (1.75, 0.25) {\(\stackrel{?}{...}\)};
                \node (4) at (2.5, 0.25) {};
                \node[label=above right:{\(t\)}] (5) at (3.5, 0.5) {};
                \node (6) at (-0.75, 0.5) {};
                \node (7) at (1.2, 1.0) {};
                \node (8) at (1.8, 1.3) {};
                \draw[->] (0) -- (1);
                \draw[->] (0) -- (2);
                \draw[->] (1) -- (3);
                \draw[->] (3) -- (4);
                \draw[->] (4) -- (5);
                \draw[->] (8) -- (7);
                \draw (1.75, 0.25) ellipse (3.0cm and 2cm);
                \node[fill=none] at (-1.25, 1.75) {\(G\)};
            \end{tikzpicture}
        \end{figure}
    \end{frame}

    \begin{frame}
        \frametitle{\(\pathprob \in \pclass\)}
        \begin{theorem}
            \(\pathprob \in \pclass\)
        \end{theorem}
        \pause
        \begin{proof}
            \begin{description}
                \item[Input:] \(\enc{G, s, t}\), where \(G\) is a directed graph with nodes \(s, t\).
                \item[Function:] \phantom{} \begin{enumerate}[label=\arabic*.]
                    \item Mark \(s\).
                    \item Repeat until no more nodes are marked:
                    \begin{enumerate}[label=2.\arabic*.]
                        \item Search through edges \(E\). If an edge \((u, v)\) is found with \(u\) marked and \(v\) not marked, mark \(v\).
                    \end{enumerate}
                    \item If \(t\) is marked, accept. Otherwise, reject.
                \end{enumerate}
            \end{description}
        \end{proof}
    \end{frame}

    \begin{frame}
        \frametitle{The RELPRIME Problem}
        If two natural integers have the greatest common divisor (GCD) 1, they are called \emph{relatively prime}. 
        \pause
        \[
            \relprimeprob \coloneqq \{ \; \enc{x, y} \mid x \text{ and } y \text{ are relatively prime} \; \}
        \]
    \end{frame}

    \begin{frame}
        \frametitle{The Euclidian Algorithm}
    
        \begin{figure}[!htbp]
            \centering
            \scriptsize
            \begin{tikzpicture}[
                scale=18/14*0.25,
                very thin
            ]
                \draw[fill=red!50] (0, 0) -- (14, 0) node[midway, above] {\(14\)} -- (14, -9) -- (0, -9) -- (0, 0) node[midway, left] {\(9\)};
                \draw[fill=green!50] (0, -9) -- (9, -9) -- (9, -14) -- (0, -14) node[midway, below] {\(9\)} -- (0, -9) node[midway, left] {\(5\)};
                \draw[fill=blue!50] (9, -9) -- (14, -9) -- (14, -13) node[midway, right] {\(4\)} -- (9, -13) node[midway, above, text=white] {\(5\)} -- (9, -9);
                \draw[fill=yellow!50] (9, -13) -- (13, -13) -- (13, -14) -- (9, -14) node[midway, below] {\(4\)} -- (9, -13) node[midway, left] {\(1\)};
                \draw[fill=gray!50] (13, -13) -- (14, -13) -- (14, -14) node[midway, right] {\(1\)} -- (13, -14) node[midway, below] {\(1\)} -- (13, -13);
            \end{tikzpicture}
            \begin{tikzpicture}[
                scale=0.25,
                very thin
            ]
                \draw[fill=green!50] (0, 0) -- (18, 0) node[midway, above] {\(18\)} -- (18, -12) -- (0, -12) -- (0, 0) node[midway, left] {\(12\)};
                \draw[fill=blue!50] (0, -12) -- (12, -12) -- (12, -18) -- (0, -18) node[midway, below] {\(12\)} -- (0, -12) node[midway, left] {\(6\)};
                \draw[fill=gray!50] (12, -12) -- (18, -12) -- (18, -18) node[midway, right] {\(6\)} -- (12, -18) node[midway, below] {\(6\)} -- (12, -12);
            \end{tikzpicture}
        \end{figure}
    
    \end{frame}

    \begin{frame}
        \frametitle{The RELPRIME Problem}

        \begin{theorem}
            \(\relprimeprob \in \pclass\)
        \end{theorem}

        \pause
    
        \begin{proof}
            \begin{description}
                \item[Input:] \(\enc{x, y}\), where \(x, y \in \mathbb{N}\).
                \item[Function:] \phantom{}
                \begin{algorithmic}[1]
                    \While{\(y > 0\)}
                        \State \(x \gets x\bmod{y}\)
                        \State Swap \(x\) and \(y\)
                    \EndWhile
                    \State If \(x = 1\), accept. Otherwise, reject.
                \end{algorithmic}
            \end{description}
        \end{proof}
    
    \end{frame}

    \begin{frame}
        \frametitle{Context-Free Languages}
    
        \begin{theorem}
            Every context-free language \(L \in \cflclass\) is in P.
        \end{theorem}

        \begin{itemize}
            \item[\(\Rightarrow\)] CYK algorithm
        \end{itemize}
    
    \end{frame}

    \begin{frame}
        \frametitle{Reminder: CYK Algorithm}

        Follows the principle of dynamic programming. Recursion from last semester: \(V[i, j] = \{\; A \in V \mid A \rightarrow^* \sigma_i...\sigma_j\;\}\)

        \begin{figure}[!hbtp]
            \centering
            \begin{tikzpicture}[
                scale=0.85
            ]
                \def\x{0};
                \foreach \y in {0, 0, 1, ..., 5}
                {
                    \draw[very thin] (\x, 0) -- (\x, \numexpr6-\y);
                    \pgfmathparse{\x+1};
                    \xdef\x{\pgfmathresult};
                }
                
                \def\y{0};
                \foreach \x in {0, 0, 1, ..., 5}
                {
                    \draw[very thin] (0, \y) -- (\numexpr6-\x, \y);
                    \pgfmathparse{\y+1};
                    \xdef\y{\pgfmathresult};
                }
        
                \draw[fill=gray!25, draw=none] (1.5, 3.25) -- (1.75, 3.25) -- (4.33, -0.25) -- (1.5, -0.25) -- cycle;
                \node at (2.25, 1) {\(\rightarrow^*\)};
                \node (A) at (1.5,  3.5) {\(A\)};
                \node (1) at (0.5, -0.5) {\(\sigma_1\)};
                \node (2) at (1.5, -0.5) {\(\sigma_2\)};
                \node (3) at (2.5, -0.5) {\(\sigma_3\)};
                \node (4) at (3.5, -0.5) {\(\sigma_4\)};
                \node (5) at (4.5, -0.5) {\(\sigma_5\)};
                \node (6) at (5.5, -0.5) {\(\sigma_6\)};
                \draw[thick] (A) -- (2);
                \draw[thick] (A) -- (5);
            \end{tikzpicture}
        \end{figure}
    \end{frame}

    \begin{frame}
        \frametitle{\(CFL \subset\pclass\)}

        \begin{proof}
            \begin{description} \footnotesize
                \item[Input:] \(w \in \Sigma^*\), for \(w \neq \varepsilon\) write \(w = \sigma_1\sigma_2...\sigma_n\).
                \item[Function:] \phantom{}
                    \begin{algorithmic}[1]
                        \State For \(w = \varepsilon\), if \(S \rightarrow \varepsilon\) is production, accept. Otherwise, reject.
                        \For{\(i = 1\) to \(n\)}
                            \For{each variable \(A\)}
                                \State If \(A \rightarrow \sigma_i\) is a rule, place \(A\) in \(table(i, i)\).
                            \EndFor
                        \EndFor
                        \For{\(l = 2\) to \(n\)} \Comment{Substring length}
                            \For{\(i = 1\) to \(n-l+1\)} \Comment{Starting position}
                                \State \(j \gets i + l - 1\) \Comment{End position}
                                \For{\(k = i\) to \(j - 1\)} \Comment{Split position}
                                    \For{each rule \(A \rightarrow BC\)}
                                        \State If \(B \in table(i, k)\) and \(C \in table(k + 1, j)\), put \(A\) in \(table(i, j)\).
                                    \EndFor
                                \EndFor
                            \EndFor
                        \EndFor
                        \State If \(S \in table(1, n)\), accept. Else, reject.
                    \end{algorithmic}
            \end{description}
        \end{proof}

    \end{frame}

    \begin{frame}
        \frametitle{Resources}
        \nocite{*}
        \renewcommand{\refname}{\normalsize References} 
        \bibliography{p_presentation}
        \bibliographystyle{plain}
    \end{frame}

    \begin{frame}
        \frametitle{Final Landscape of Languages}
        \begin{figure}[!htbp]
            \begin{tikzpicture}[
                semithick,
                scale=0.5
            ]
                \draw (-6.5, -4.5) rectangle (6.5, 5);
                \draw (0.0, 0.0) ellipse (6cm and 4cm);
                \draw (0.0, -1.5) ellipse (2cm and 1cm);
        
                \node at (0.0, 4.5) {Decidable};
                \node at (0.0, 3.5) {\pclass};
                \node at (-3.25, 0.5) {\pathprob};
                \node at (2.0, 1.0) {\relprimeprob};
                \node at (0.0, -1.5) {\cflclass};
            \end{tikzpicture}
        \end{figure}

        \pause
    
        \begin{center}
            \Large Questions?
        \end{center}
    \end{frame}
\end{document}
