\documentclass[notheorems]{beamer}

\usepackage{algorithm}
\usepackage[noend]{algpseudocode}
\usepackage{amsfonts}
\usepackage{amsmath}
\usepackage{amssymb}
\usepackage{amsthm}
\usepackage[english, ngerman]{babel}
%\usepackage{enumitem}
\usepackage{graphicx}
\usepackage{mathtools}
\usepackage{nccmath}
\usepackage{subcaption}
\usepackage{tikz}
\usetikzlibrary{arrows, calc, shapes}

\usetheme{metropolis}

\setlength{\parindent}{0pt}

% I do not really like the plain style
%\theoremstyle{plain}
\theoremstyle{definition}
\newtheorem{definition}{Definition}
\newtheorem{theorem}{Theorem}
\newtheorem{proposition}{Proposition}
\newtheorem{lemma}[theorem]{Lemma}
\newtheorem{corollary}{Corollary}
\newtheorem{fact}{Fact}
\theoremstyle{remark}
\newtheorem{remark}{Remark}

\newcommand{\pclass}{\text{P}}
\newcommand{\npclass}{\text{NP}}
\newcommand{\pathprob}{\text{PATH}}
\newcommand{\relprimeprob}{\text{RELPRIME}}
\newcommand{\cflclass}{\text{CFL}}
\newcommand{\hampathprob}{\text{HAMPATH}}
\newcommand{\compositesprob}{\text{COMPOSITES}}
\newcommand{\cliqueprob}{\text{CLIQUE}}
\newcommand{\subsetsumprob}{\text{SUBSET-SUM}}
\newcommand{\satprob}{\text{SAT}}
\newcommand{\threesatprob}{\text{3SAT}}
\newcommand{\spaceclass}{\text{SPACE}}
\newcommand{\nspaceclass}{\text{NSPACE}}
\newcommand{\pspaceclass}{\text{PSPACE}}
\newcommand{\npspaceclass}{\text{NPSPACE}}
\newcommand{\exptimeclass}{\text{EXPTIME}}
\newcommand{\tqbfprob}{\text{TQBF}}
\newcommand{\formulagameprob}{\text{FORMULA-GAME}}
\newcommand{\ggprob}{\text{GG}}

\newcommand{\lpp}{\left \langle}
\newcommand{\rpp}{\right \rangle}
\newcommand{\enc}[1]{\lpp #1 \rpp}

\DeclareMathOperator{\onot}{\mathcal{O}}
\DeclareMathOperator{\omnot}{\Omega}
\DeclareMathOperator{\thetnot}{\Theta}

\renewcommand{\qedsymbol}{\(\blacksquare\)}

\title{The Class PSPACE and Game Theory}
\author{valentinpi}
\institute{\normalsize Proseminar Theoretische Informatik WiSe 2020-21\\
Institut für Informatik\\
Freie Universität Berlin}
\date{5. Januar 2021{ }(neueste Version)}

\begin{document}
\selectlanguage{ngerman}
\frame{\titlepage}
\selectlanguage{english}

\begin{frame}
    \frametitle{Space Complexity}
    \pause
    \begin{definition} Let \(f\colon \mathbb{N} \rightarrow \mathbb{R}^+\) be a function. The complexity classes \emph{SPACE} and \emph{NSPACE} of \(f\) are:
        \pause
        \begin{align*}
            \spaceclass(f(n)) \coloneqq \{ \; L \mid &\; L \text{ is decided by a deterministic turing machine}\\&\text{ using } \mathcal{O}(f(n)) \text{ space} \; \}
        \end{align*}
        \pause
        \vspace{-2\baselineskip}
        \begin{align*}
            \nspaceclass(f(n)) \coloneqq \{ \; L \mid &\;L \text{ is decided by a nondeterministic turing}\\&\text{ machine using } \mathcal{O}(f(n)) \text{ space} \; \}
        \end{align*}
    \end{definition}
\end{frame}

\begin{frame}
    \frametitle{PSPACE and NPSPACE}
    \pause
    \begin{definition} In analogy to the classes P and NP, \emph{PSPACE} and \emph{NPSPACE} are defined as:
        \[
            \pspaceclass \coloneqq \bigcup_{k \in \mathbb{N}} \spaceclass(n^k) \qquad
            \npspaceclass \coloneqq \bigcup_{k \in \mathbb{N}} \nspaceclass(n^k)
        \]
    \end{definition}
\end{frame}

\begin{frame}
    \frametitle{Another inequality?}

    Relations between PSPACE and NPSPACE?

    \pause

    Surprising result (Savitch):
    
    \begin{fact}[Corollary from Savitchs Theorem]
        \(
            \pspaceclass = \npspaceclass
        \)
    \end{fact}

\end{frame}

\begin{frame}
    \frametitle{PSPACE-Completeness}

    \pause

    \begin{definition} A language \(B\) is \emph{PSPACE-complete}, if:
        \begin{enumerate}
            \item \(B \in \pspaceclass\)
            \item \(A \leq_p B\) for all \(A \in \pspaceclass\), also called \emph{PSPACE-hard}
        \end{enumerate}
    \end{definition}

\end{frame}

\begin{frame}
    \frametitle{Effects on the Landscape}

    \begin{fact} The following holds:
        \pause
        \begin{ceqn}
            \begin{align*}
                &\pclass \subseteq \npclass \subseteq \pspaceclass = \npspaceclass \subseteq \exptimeclass \coloneqq \bigcup_{k \in \mathbb{N}} \text{TIME}\left(2^{n^k}\right)\\
                &\pclass \neq \exptimeclass
            \end{align*}
        \end{ceqn}
    \end{fact}

\end{frame}

\begin{frame}
    \frametitle{Quantified Boolean Formulas}

    \pause

    Generalization of SAT.
    
    Prerequisites:
    \begin{itemize}
        \pause
        \item \emph{Prenex Normal Form}: \(\text{Q}_1x_1\text{Q}_2x_2...\text{Q}_nx_n\colon\phi\) with \(\text{Q}_1,...,\text{Q}_n \in \{\forall,\exists\}\)
        \pause
        \item \emph{Fully Quantified Boolean Formulas (Statements)}
    \end{itemize}

\end{frame}

\begin{frame}
    \frametitle{The TQBF Problem}

    \begin{ceqn}
        \[
            \tqbfprob \coloneqq \{\; \enc{\phi} \mid \phi \text{ is a true fully quantified boolean formula}\;\}
        \]
    \end{ceqn}

\end{frame}

\begin{frame}
    \frametitle{TQBF is PSPACE-Complete}

    \begin{theorem}
        \(
            \tqbfprob
        \) is PSPACE-complete.
    \end{theorem}
    \begin{proof}
        \pause
        \(\tqbfprob \in \pspaceclass\):
        \begin{description}
            \scriptsize
            \item[Input:] \(\enc{\phi}\), where \(\phi\) is a fully quantified boolean formula.
            \item[Function:] \phantom{}
                \begin{algorithmic}[1]
                    \If{\(\phi = \psi\), no quantifiers}
                        \State Evaluate \(\psi\), accept or reject depending on result.
                    \ElsIf{\(\phi = \exists x\colon \psi\)}
                        \State Recursive call on \(\psi\) with \(x=\text{true}, x=\text{false}\).
                        \State If either accepts, accept. Otherwise, reject.
                    \Else{ \(\phi = \forall x\colon \psi\)}
                        \State Recursive call on \(\psi\) with \(x=\text{true}, x=\text{false}\).
                        \State If both accept, accept. Otherwise, reject.
                    \EndIf
                \end{algorithmic}
        \end{description}

        \renewcommand{\qedsymbol}{}
    \end{proof}

\end{frame}

\begin{frame}
    \frametitle{TQBF is PSPACE-Complete}

    \begin{theorem}
        \(
            \tqbfprob
        \) is PSPACE-complete.
    \end{theorem}
    \begin{proof}
        \pause
        TQBF is PSPACE-hard. (Proof idea)

        Use Cook-Levin?

        \[
            \text{Machine} \longmapsto \text{Fully Quantified Boolean Formula}
        \]

        \pause
            
        Machine may run in exponential time \(\Rightarrow\) Exponential sized result

        \pause

        \(\Rightarrow\) Shorten Formula using \(\forall\) (Savitchs Theorem)
    \end{proof}

\end{frame}

\begin{frame}
    \frametitle{Proving more PSPACE-Completeness}

    From NP-completeness:
    \pause
    \begin{corollary}
        If \(B\) is PSPACE-complete and \(B \leq_p C\) for \(C \in \pspaceclass\), then \(C\) is PSPACE-complete.
    \end{corollary}

\end{frame}

\begin{frame}
    \frametitle{Games}

    \begin{ceqn}
        \[
            \underbrace{\text{Player I} \rightarrow \text{Goals} \leftarrow \text{Player II}}_{\text{Rules}}
        \]
    \end{ceqn}
    \pause
    Game: \emph{Formula game} between players A and E.

    \pause
    Want: \emph{Winning strategy}

\end{frame}

\begin{frame}
    \frametitle{Equality of Problems}

    \pause

    \begin{corollary} \(\formulagameprob = \tqbfprob\) and \(\formulagameprob\) is PSPACE-complete.
    \end{corollary}

\end{frame}

\begin{frame}
    \frametitle{Geography and the World}

\end{frame}

\begin{frame}
    \frametitle{A Game of Geography on France}

    \begin{figure}[!hbtp]
        \centering
        \begin{subfigure}{0.8\linewidth}
            \resizebox{\linewidth}{!}{%
                \begin{tikzpicture}[
                    ->,
                    >=stealth,
                    every node/.style={draw, ellipse}
                ]
                    \scriptsize
                    \node (0) at (0, 0) {Paris};
                    \node (1) at (-2, -1) {St. Malo};
                    \node (2) at (1, -1.5) {Orleans};
                    \node (3) at (4, -1) {Strasbourg};
                    \node (4) at (3, -3) {Grenoble};
                    \node (5) at (2, 0) {Epernay};
                    \path (-1, 0) edge (0)
                    (0) edge[bend left=25] node[draw=none, right] {I} (1)
                    (1) edge node[draw=none, below] {II} (2)
                    (2) edge node[draw=none, above right] {I} (3)
                    (3) edge node[draw=none, above left] {II} (4)
                    (4) edge node[draw=none, below left] {I} (5);
                \end{tikzpicture}%
            }
        \end{subfigure}
    \end{figure}

\end{frame}

\begin{frame}
    \frametitle{Generalized Geography}

    Generalize for any directed graph:
    
    \begin{ceqn}
        \begin{align*}
            \ggprob \coloneqq \{\; \enc{G, s} \mid &\;\text{Player I has a winning strategy for the GG game}\\&\text{ played on }G\text{, starting from } s \;\}
        \end{align*}
    \end{ceqn}

\end{frame}

\begin{frame}
    \frametitle{GG is PSPACE-Complete}

    \begin{theorem}
        \(\ggprob\) is PSPACE-complete.
    \end{theorem}

    \begin{proof} \(\formulagameprob \leq_p \ggprob\)

        \(\ggprob \in \pspaceclass\)
        \begin{description}
            \scriptsize
            \item[Input:] \(\enc{G, s}\), where \(G\) is a directed graph and \(s\) a node from \(G\).
            \item[Function:] \phantom{}
                \begin{algorithmic}[1]
                    \If{\(\deg{s} = 0\)}
                        \State Reject, since Player I instantly looses.
                    \EndIf
                    \State Remove \(s\) and its edges from \(G\) for a new graph \(G'\).
                    \State For the outgoing neighbors of \(s\), \(s_1, ..., s_k\), do a recursive call on \(\enc{G', s_i}\), \(i \in \{1,...,k\}\).
                    \State If all accept, reject. Otherwise, accept.
                \end{algorithmic}
        \end{description}
        \renewcommand{\qedsymbol}{}
    \end{proof}

\end{frame}

\begin{frame}
    \frametitle{GG is PSPACE-Complete}

    \begin{theorem}
        \(\ggprob\) is PSPACE-complete.
    \end{theorem}

    \begin{proof} \(\formulagameprob \leq_p \ggprob\)

        \[\phi = \exists x_1 \forall x_2 \exists x_3 ... \exists x_n\colon \psi \mapsto (G, s)\]
        \renewcommand{\qedsymbol}{}

    \end{proof}

\end{frame}

\begin{frame}[plain]
    \begin{tikzpicture}[
        every node/.style={draw, circle, minimum size=5mm, inner sep=0mm},
        >=stealth,
        scale=1
    ]
        \footnotesize
        \node (0) at (0, 0) {\(s\)};
        \node[below left of=0] (10) {};
        \node[below right of=0] (11) {};
        \node[draw=none, left of=10] {\(x_1\)};
        \node[below right of=10] (2) {};
        \node[below of=2] (3) {};
        \node[below left of=3] (40) {};
        \node[below right of=3] (41) {};
        \node[draw=none, left of=40] {\(x_2\)};
        \node[below right of=40] (5) {};
        \node[below of=5] (6) {};
        \node[draw=none,below of=6] (7) {...};
        \node[below of=7] (8) {};
        \node[below left of=8] (90) {};
        \node[below right of=8] (91) {};
        \node[draw=none, left of=90] {\(x_n\)};
        \node[below right of=90] (100) {};
        \path[->, draw]
        (0) edge node[above left, draw=none] {\scriptsize true} (10)
        (0) edge node[above right, draw=none] {\scriptsize false} (11)
        (10) edge (2)
        (11) edge (2)
        (2) edge (3)
        (3) edge (40)
        (3) edge (41)
        (40) edge (5)
        (41) edge (5)
        (5) edge (6)
        (6) edge (7)
        (7) edge (8)
        (8) edge (90)
        (8) edge (91)
        (90) edge (100)
        (91) edge (100)
        ;
        \pause
        \node (c) at (6, -4) {\(c\)};
        \node (c1) at (4, -1) {\(c_1\)};
        \node (c2) at (4, -3) {\(c_2\)};
        \node[draw=none] (etc) at (4, -5) {...};
        \node (cm) at (4, -7) {\(c_m\)};
        \node (110) at (3, -0.33) {\(x_1\)};
        \node (111) at (3, -1) {\(\overline{x_2}\)};
        \node (112) at (3, -1.66) {\(x_3\)};
        \node (120) at (3, -2.33) {\(\overline{x_1}\)};
        \node (121) at (3, -3) {\(x_2\)};
        \node (122) at (3, -3.66) {};
        \node (130) at (3, -6.33) {};
        \node (131) at (3, -7) {};
        \node (132) at (3, -7.66) {};
        \path[->, draw]
        (100) edge[bend right=60] (c)
        (c) edge (c1)
        (c) edge (c2)
        (c) edge (cm)
        (c1) edge (110)
        (c1) edge (111)
        (c1) edge (112)
        (c2) edge (120)
        (c2) edge (121)
        (c2) edge (122)
        (cm) edge (130)
        (cm) edge (131)
        (cm) edge (132)
        ;
        \pause
        \path[->, draw]
        (110) edge[bend right=10] (10)
        (111) edge (41)
        (120) edge (11)
        (121) edge[bend left=25] (40)
        ;
    \end{tikzpicture}
\end{frame}

\begin{frame}
    \frametitle{Resources}
    \nocite{*}
    \renewcommand{\refname}{\normalsize References}
    \bibliography{pspace_presentation}
    \bibliographystyle{plain}
\end{frame}

\begin{frame}
    \frametitle{Another New Landscape}

    \pause
    \begin{figure}[!hbtp]
        \large
        \centering
        \begin{tikzpicture}[
            semithick,
            scale=0.75
        ]
            \draw (-4.5, -1.5) rectangle (6.5, 6.5);
            \node at (-3, 5.75) {EXPTIME};
            \draw (0, 0) ellipse (0.5cm and 0.5cm);
            \node at (0, 0) {P};
            \draw (0.5, 0.25) ellipse (1.5cm and 1cm);
            \node at (1, 0.5) {NP};
            \node at (1, 5) {\(\pspaceclass = \npspaceclass\)};
            \draw (1, 2.5) ellipse (5cm and 3.5cm);
            \node at (1, 3.5) {\normalsize PSPACE-complete};
            \node at (-0.5, 2) {\normalsize \(\ggprob\)};
            \node at (1.75, 2.6) {\normalsize \(\begin{matrix}
                &\tqbfprob =\\
                &\formulagameprob
            \end{matrix}\)};
            \draw (1.25, 2.75) ellipse (3.15cm and 1.4cm);
        \end{tikzpicture}
    \end{figure}
    (Assuming relationships are proper)

    \pause

    \begin{center}
        \Large Questions?
    \end{center}

\end{frame}

\end{document}
